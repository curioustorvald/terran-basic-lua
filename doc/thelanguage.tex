This chapter describes the lexis, the syntax, and the semantics of TBASIC. In other words, this chapter describes which words are good, how they can be jumbled up, and what the hell they could even mean. 

\section{Line Format}

Program lines in BASIC program have the following format:

\forceindent{\monofont nnnnn BASIC-statement <carriage return>} \

Only one BASIC statement may be placed on a line. A program line always begins with a line number and ends with a carriage return.


\section{Lexical Conventions}

\emph{Names} (also called \emph{identifiers}) in TBASIC can be any string of letters, digits, and underscores, not beginning with a digit. Identifiers are used to name variables.

The following \emph{keywords} are reserved and cannot be used as names:

\begin{multicols}{6}
%{\condensedfont%
ABORT\par
ABORTM\par
ABS\par
AND\par %operator
ASC\par
BACKCOL\par
BEEP\par
CBRT\par
CEIL\par
CHR\par
CLR\par
CLS\par
COS\par
DEF\par
DIM\par
DO\par
END\par
FALSE\par
FLOOR\par
FN\par
FOR\par
GET\par
GO\par
GOSUB\par
GOTO\par
HTAB\par
IF\par
INPUT\par
INT\par
INV\par
LEFT\par
LEN\par
LIST\par
LOAD\par
LOG\par
M\_E\par
M\_PI\par
M\_2PI\par
M\_ROOT2\par
MAX\par
MID\par
MIN\par
MINUS\par %operator
NEW\par
NEXT\par
NIL\par
NOT\par %operator
OR\par %operator
PRINT\par
REM\par
RETURN\par
RIGHT\par
RND\par
ROUND\par
RUN\par
SAVE\par
SCROLL\par
SGN\par
SIN\par
SIZEOF\par %operator
SQRT\par
STEP\par %operator
STR\par
TAB\par
TAN\par
TEXTCOL\par
THEN\par
TO\par %operator
TEMIT\par
TRUE\par
VAL\par
VTAB\par
XOR\par %operator
%}%
\end{multicols}

TBASIC is a case-insensitive language, and so is the reserved word.

The following strings denote other tokens:

\begin{multicols}{6}
%{\condensedfont%
$>$$>$$>$\par
$<$$<$\par
$>$$>$\par
|\par
\&\par
!\par
;\par
$=$$=$\par
$>$\par
$<$\par
$<$$=$\par
$=$$<$\par
$>$$=$\par
$=$$>$\par
!$=$\par
$<$$>$\par
$>$$<$\par
$=$\par
$:$$=$\par
$^\wedge$\par
*\par
/\par
$+$\par
$-$\par
\%\par
(\par
)\par
,\par
%}
\end{multicols}


\section{Constants}

TBASIC predefines some constants using its variable system. You can read from these variables, but any new assignment will have no effects and TBASIC will \emph{not} raise any errors.

\begin{tabularx}{\textwidth}{l l X}
	\textbf{Name} & \textbf{Value} & \textbf{Remarks}
	\\
	\endhead
	M\_PI    & 3.141592653589 & Mathematical constant $\pi$ \\
	M\_2PI   & 6.283185307180 & Mathematical constant $2\pi$ \\
	M\_E     & 2.718281828459 & Mathematical constant $e$ \\
	M\_ROOT2 & 1.414213562373 & Mathematical constant $\sqrt{2}$ \\
	TRUE    & \emph{true} & Self explanatory \\
	FALSE   & \emph{false} & do. \\
	NIL     & \emph{nil} & do. \\
	\_VERSION & \tbasver & Current TBASIC version \\
\end{tabularx}
