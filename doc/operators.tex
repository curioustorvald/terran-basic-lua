This chapter describes operators that can be used in TBASIC.

\section{Overview}

There are 33 operators in TBASIC version \tbasver.

\section{Precedence}

Operators has something called \emph{precedence}. It's not something hard despite of its name (especially if you are not a native English speaker, like me). It's just a thing that describes our language programmatically\footnote{in computer language}, which goes like \emph{multiplication comes before the addition and subtraction}.

\begin{tabularx}{\textwidth}{c X}
	\textbf{Order} & \textbf{Operators}
	\\
	\endhead
	1 & :$=$\quad $=$ \\
	2 & OR \\
	3 & AND \\
	4 & | \\
	5 & XOR \\
	6 & \& \\
	7 & $=$$=$\quad !$=$\quad $<$$>$\quad $>$$<$ \\
	8 & $<$$=$\quad $>$$=$\quad $=$$<$\quad $=$$>$\quad $<$\quad $>$ \\
	9 & TO\quad STEP \\
	10 & $>$$>$$>$\quad $<$$<$\quad $>$$>$ \\
	11 & ; \\
	12 & $+$\quad $-$ \\
	13 & *\quad /\quad \% \\
	14 & NOT\quad ! \\
	15 & $^\wedge$\quad SIZEOF \\
	16 & MINUS \\
\end{tabularx}

\section{Precautions}

Because of the way the language is desingned\footnote{don't write like \monobf{1+3*2+4/7}, they look hideous}, you must be careful when using $ - $ operator.

\begin{tabularx}{\textwidth}{l l X}
	\textbf{Code} & \textbf{Evaluation} & \textbf{Remarks}
	\\
	\endhead
	\mono{FOO - 1} & subtraction(FOO, 1) & Works normally
	\\
	\mono{FOO-1} & FOOMINUS1 & Syntax error
\end{tabularx}


\section{Arithmetic Operators}

\subsection{PLUS} \emph{number} $+$ \emph{number} --- adds two numbers
\subsection{MINUS} \emph{number} $-$ \emph{number} --- subtracts two numbers
\subsection{TIMES} \emph{number} * \emph{number} --- multiplies two numbers
\subsection{DIVIDE} \emph{number} / \emph{number} --- divides two numbers
\subsection{MODULO} \emph{number} \% \emph{number} --- gets modulo (remainder) of two numbers
\subsection{POWEROF} \emph{number a} $^\wedge$ \emph{number b} --- gets \emph{a} to the power of \emph{b}, or $a ^b$


\section{Logical Operators}

\subsection{AND} \emph{condition 1} AND \emph{condition 2} --- returns \textbf{TRUE} if both contidions (boolean value) are true; false otherwise
\subsection{OR} \emph{condition 1} OR \emph{condition 2} --- returns \textbf{TRUE} if one or more conditions (boolean value) are true; \textbf{FALSE} if both conditions are false
\subsection{NOT} NOT \emph{condition} --- negates condition (boolean value)


\section{Relational Operators}

\subsection{EQUALTO} \emph{number} $=$$=$ \emph{number} --- returns \textbf{TRUE} if two numbers represent same quantity; \textbf{FALSE} otherwise
\subsection{NOTEQUALTO} \emph{number} !$=$ \emph{number} --- returns \textbf{TRUE} if two numbers represent different quantity; \textbf{FALSE} otherwise; aliases: $<$$>$, $>$$<$
\subsection{GREATERTHAN} \emph{number a} $>$ \emph{number b} --- returns \textbf{TRUE} if $a > b$ is satisfied; \textbf{FALSE} otherwise
\subsection{LESSTHAN} \emph{number a} $<$ \emph{number b} --- returns \textbf{TRUE} if $a < b$ is satisfied; \textbf{FALSE} otherwise
\subsection{GEQTHAN} \emph{number a} $>$$=$ \emph{number b} --- returns \textbf{TRUE} if $a \geq b$ is satisfied; \textbf{FALSE} otherwise; alias: $=$$<$
\subsection{LEQTHAN} \emph{number a} $<$$=$ \emph{number b} --- returns \textbf{TRUE} if $a \leq b$ is satisfied; \textbf{FALSE} otherwise; alias: $=$$>$


\section{Assignment Operators}

\subsection{AS} \emph{variable} $=$ \emph{value} --- assign \emph{value} to \emph{variable}; alias: :$=$


\section{Bitwise Operators}

\subsection{BAND} \emph{integer} \& \emph{integer} --- performs bitwise AND on two integers
\subsection{BOR} \emph{integer} | \emph{integer} --- performs bitwise OR on two integers
\subsection{BNOT} ! \emph{integer} --- performs bitwise NOT on integer
\subsection{BXOR} \emph{integer} XOR \emph{integer} --- performs bitwise XOR on two integers
\subsection{LSHIFT} \emph{integer} $<$$<$ \emph{offset} --- performs left shift on \emph{integer} by \emph{offset}
\subsection{RSHIFT} \emph{integer} $>$$>$ \emph{offset} --- performs signed right shift on \emph{integer} by \emph{offset}; if leftmost bit is 1, bits on the left side are padded with 1
\subsection{URSHIFT} \emph{integer} $>$$>$$>$ \emph{offset} --- performs unsigned right shift on \emph{integer} by \emph{offset}


\section{Miscellaneous Operators}

\subsection{CONCAT} \emph{string} ; \emph{string} --- concatenates two strings
\subsection{MINUS} $-$\emph{number} --- negates \emph{number}, or $-n$
\subsection{SIZEOF} SIZEOF \emph{array} --- returns size of array
\subsection{TO} \emph{number} TO \emph{number} --- creates integer sequence of specified range; \mono{1 TO 4} will create an array of \emph{1, 2, 3, 4}; \mono{8 TO 3} will create \emph{8, 7, 6, 5, 4, 3}
\subsection{STEP} \emph{int array} STEP \emph{number} --- filters input array such that every $1+n$th number is remained; \mono{1 TO 10 STEP 3} will create an array of {1, 4, 7, 10}