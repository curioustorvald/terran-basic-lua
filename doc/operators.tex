This chapter describes operators that can be used in TBASIC.

\section{Overview}

There are 33 operators in TBASIC version \tbasver.

\section{Precedence}

Operators has something called \emph{precedence}. It's not something hard despite of its name (especially if you are not a native English speaker, like me). It's just a thing that describes our language programmatically\footnote{in computer language}, which goes like \emph{multiplication comes before the addition and subtraction}.

\begin{tabularx}{\textwidth}{c X}
	\textbf{Order} & \textbf{Operators}
	\\
	\endhead
	1 & $>$$>$$>$\quad $<$$<$\quad $>$$>$\quad |\quad \&\quad XOR\quad ! \\ 
	2 & ; \\ 
	3 & SIZEOF \\ 
	4 & $=$$=$\quad $>$\quad $<$\quad $<$$=$\quad $=$$<$\quad $>$$=$\quad $=$$>$\quad  \\ 
	5 & !$=$\quad $<$$>$\quad $>$$<$ \\ 
	6 & $=$\quad :$=$ \\ 
	7 & AND\quad OR\quad NOT \\ 
	8 & $^\wedge$ \\ 
	9 & *\quad /\quad $+$\quad $-$ \\ 
	10 & \% \\ 
	11 & TO\quad STEP \\ 
	12 & MINUS
\end{tabularx}

\section{Precautions}

Because of the way the language is desingned\footnote{don't write like \monobf{1+3*2+4/7}, they look hideous}, you must be careful when using $ - $ operator.

\begin{tabularx}{\textwidth}{l l X}
	\textbf{Code} & \textbf{Evaluation} & \textbf{Remarks}
	\\
	\endhead
	\mono{FOO - 1} & subtraction(FOO, 1) & Works normally
	\\
	\mono{FOO-1} & FOOMINUS1 & Syntax error
\end{tabularx}

\section{Synopsis}

\subsection{SIZEOF} \emph{array} returns size of the array